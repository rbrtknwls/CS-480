\documentclass{article}
\usepackage[margin=1in]{geometry}
\usepackage{amsmath,amsfonts,amssymb}
\usepackage{listings}
\usepackage{color}
\usepackage{graphicx}
\usepackage{subfig}
\usepackage{blkarray}
\usepackage{multirow}
\usepackage{float}
\usepackage{caption}
\usepackage{subcaption}
\begin{document}
\begin{titlepage}
	\setlength{\parindent}{0pt}
	\large

\vspace*{-2cm}

\definecolor{dkgreen}{rgb}{0,0.6,0}
\definecolor{gray}{rgb}{0.5,0.5,0.5}
\definecolor{mauve}{rgb}{0.58,0,0.82}

\lstset{frame=tb,
  language=Python,
  aboveskip=3mm,
  belowskip=3mm,
  showstringspaces=false,
  columns=flexible,
  basicstyle={\small\ttfamily},
  numbers=none,
  numberstyle=\tiny\color{gray},
  keywordstyle=\color{blue},
  commentstyle=\color{dkgreen},
  stringstyle=\color{mauve},
  breaklines=true,
  breakatwhitespace=true,
  tabsize=3
}

University of Waterloo \par
CS 480 \par
\vspace{0.05cm}
r2knowle: 2023-11-13
\vspace{0.2cm}

{\huge Exercise \# 1 \par}
\hrule

\vspace{0.5cm}
\textbf{Q1a)} Our goal will be to prove that regardless of the updates, $p_i$ will remain constant, as if this is the case it implies the updates are equivalent. Note that this means we need to prove for all $w_i$ and $\widetilde{w_i}$ is must be the case that:
\[ p_i^t = \frac{w_i^t}{\sum_j w_j^t} = \frac{\widetilde{w}^t_i}{\sum_j \widetilde{w_j}} \]
Further more this implies that we need to prove $w_i = a\widetilde{w_i}$ where $\exists a \in \mathbb{R}$  as if this is true it follows that:
\begin{align*}
\frac{w_i^t}{\sum_j w_j^t} &= \frac{a}{a} \times \frac{w_i^t}{\sum_j w_j^t} \\
&= \frac{aw_i^t}{\sum_j aw_j^t} \\
&= \frac{\widetilde{w}_i^t}{\sum_j \widetilde{w}_j^t}
\end{align*}
Therefore using induction we will prove that for any t that: $w_i^t = a\widetilde{w}_i^t$. \\\\
\textbf{Basecase:} At time t = 1 we initialize the weights to be $[\frac{1}{n}, ..., \frac{1}{n}] \in \mathbb{R}$ for both $w_i$ and $\widetilde{w_i}$ therefore it follows that $w_i = a \widetilde{w_i}$ for the constant a = 1, satisfying our basecase.\\\\
\textbf{Inductive Hypothesis:} We are going to assume that for any $t\geq 0, w_i^t = a \widetilde{w}^t_i$, we are now going to prove given the definitions in the question that $w_i^{t+1} = a \widetilde{w}^{t_i+1}$.\\\\
\textbf{Inductive Step:} To prove that $w_i^{t+1} = a \widetilde{w}^{t_i+1}$, we will show that $w_i^{t+1}$ is equivlant to $a \widetilde{w}^{t_i+1}$ given our hypothesis where $\exists a \in \mathbb{R}$. We will first simplify $w_i^{t+1}$ to get:
\begin{align*}
w_i^{t+1} &= w_i^t \text{exp(}-y_i\beta_th_t(\textbf{x}_i)) \\
&= w_i^t \text{exp(}-y_i(\frac{1}{2}\log{\frac{1-\epsilon_t}{\epsilon_t}})  h_t(\textbf{x}_i)) \\
&= w_i^t \text{exp(}-y_ih_t(\textbf{x}_i)\frac{1}{2}\log{\frac{1-\epsilon_t}{\epsilon_t}}) \\
&= w_i^t \text{exp(}\log{((\frac{1-\epsilon_t}{\epsilon_t}})^{-y_ih_t(\textbf{x}_i)\frac{1}{2}})) \\
&= w_i^t \frac{1-\epsilon_t}{\epsilon_t}^{-y_ih_t(\textbf{x}_i)\frac{1}{2}} \\
&= w_i^t \frac{\epsilon_t}{1-\epsilon_t}^{y_ih_t(\textbf{x}_i)\frac{1}{2}} 
\end{align*}
\newpage
Note that $\epsilon_t$ is constant for both types of update. Before we continue we also need to mention that $y_i$ and $h_i$ are only ever $1$ or $-1$. This means that if they are the same value $y_ih_i$ = 1 and -1 if otherwise. \\\\
It thus follows that $1- | h_t(x_i) -y_i|$ will be equivalent, as if they are the same the equation will be 1, and if they are different its - 1. Therefore we can replace our equation to get:
\begin{align*}
&= w_i^t \widetilde{\beta}_t^{(1- | h_t(x_i) - y_i |)\frac{1}{2}} \\ 
&= w_i^t \widetilde{\beta}_t^{(1- | (2\widetilde{h}_t(x_i)  -1) - (2\widetilde{y}_i - 1)|)\frac{1}{2}}  \\
&= w_i^t \widetilde{\beta}_t^{\frac{1}{2} -|\widetilde{y}_i - \widetilde{h}_t(x_i)|}\\
&= w_i^t \widetilde{\beta}_t^{1 -|\widetilde{y}_i - \widetilde{h}_t(x_i)|}\widetilde{\beta}_t^{-\frac{1}{2}} \\
&= a \widetilde{w}_i^t \widetilde{\beta}_t^{1 -|\widetilde{y}_i - \widetilde{h}_t(x_i)|}\widetilde{\beta}_t^{-\frac{1}{2}}
\end{align*}
Note that since a just needs to be a constant for each $w_i$ but not for each iteration and $\beta_t$ is a constant as $\epsilon_t$ is a constant, we will set a new $a_t = a_{t-1}\times\widetilde{\beta}_t^{-\frac{1}{2}}$ and thus we get:
\begin{align*}
&= a_{t-1} \widetilde{w}_i^t \widetilde{\beta}_t^{1 -|\widetilde{y}_i - \widetilde{h}_t(x_i)|}\widetilde{\beta}_t^{-\frac{1}{2}} \\
w_i^{t+1}&= a_t \widetilde{w}_i^{t} \widetilde{\beta}_t^{1 -|\widetilde{y}_i - \widetilde{h}_t(x_i)|} \\
w_i^{t+1}&= a \widetilde{w}_i^{t+1}
\end{align*}
Thus proving the induction, and showing how $\forall t \geq 0, w_i^{t} = a_t \widetilde{w}_i^{t}$. Which thus means $p_i$ will be the same for both updates, and therefore both equations are equivalent.\\\\
\textbf{Q1b)} It then follows from our definition of expected values:
\[ E[e^{-yH} |X=x]  = e^{-H}\times Pr(y=1 | X=x) + e^{H}\times Pr(y=-1 | X=x) \]
If we take the derivative w.r.t to H we thus get:
\[ -e^{-H}\times Pr(y=1 | X=x) + e^{H}\times Pr(y=-1 | X=x) = 0 \]
We can then rearrange to get:
\begin{align*}
-e^{-H}\times Pr(y=1 | X=x) &=  -e^{H}\times Pr(y=-1 | X=x)  \\
\frac{Pr[ y = 1 | X = x ]}{Pr[ y = -1 | X = x ]} &= \frac{-e^H}{-e^{-H}} \\
&= e^{2H} \\
\end{align*}
Now we will take the log of both sides to get:
\begin{align*}
\frac{Pr[ y = 1 | X = x ]}{Pr[ y = -1 | X = x ]} &= e^{2H} \\
\log\frac{Pr[ y = 1 | X = x ]}{Pr[ y = -1 | X = x ]} &= \log(e^{2H}) \\
&=  2H \\
&\propto H \\
&= \sum_i^T \beta_th_t(x)
\end{align*}
Thus proving how that minimizer of the given exponential loss is proportional to the log odd loss as required.\\\\
\textbf{Q1c)} We are given the definition that:
\[ \epsilon_t = \epsilon_t(h_t(x)) = \sum_{i=1}^n p^t_i \times [[h_t(x_i) \neq y_i ]]  \]
Note that since $h_i$ and $y_i$ is either -1 or 1, this would imply that $[[-h_t(x_i) \neq y_i ]]$ is the same as $[[h_t(x_i) = y_i ]]$. Therefore we can define $\widetilde{\epsilon_t}$ to be:
\[ \widetilde{\epsilon}_t = \widetilde{\epsilon}_t(-h_t(x)) = \sum_{i=1}^n p^t_i \times [[h_t(x_i) = y_i ]] \]
Notice that the sum of $\widetilde{\epsilon_t}$ and $\epsilon_t$ has the following property:
\begin{align*}
\epsilon_t + \widetilde{\epsilon_t} &= \sum_{i=1}^n p^t_i \times [[h_t(x_i) \neq y_i ]] + \sum_{i=1}^n p^t_i \times [[h_t(x_i) = y_i ]] \\
&= \sum_{i=1}^n p^t_i \\
&= \sum_{i=1}^n \frac{w_i^t}{\sum_{j=1}^n w_j^t}\\
&= \frac{\sum_{i=1}^nw_i^t}{\sum_{j=1}^n w_j^t} \\
&= 1
\end{align*}
Which thus implies that:
\[ \epsilon_t = 1 - \widetilde{\epsilon}_t \]
\[ \widetilde{\epsilon}_t = 1 - \epsilon_t \]
\newpage
Moving forward we can then define a new $\widetilde{\beta}_t$ based on the $\widetilde{\epsilon}_t$:
\begin{align*}
\widetilde{\beta}_t = \frac{1}{2} \log \frac{1-\widetilde{\epsilon}_t}{\widetilde{\epsilon}_t}
\end{align*}
To prove the updates to $w_t$ are equivalent we will prove that:
\[ w^t_i \text{exp}(-y_i\beta_th_t(x)) = w^t_i \text{exp}(-y_i\widetilde{\beta}_t\widetilde{h}_t(x)) \]
To do so we will begin with:
\begin{align*}
w^t_i \text{exp}(-y_i\beta_th_t(x)) &= w^t_i \text{exp}(y_i\beta_t\widetilde{h}_t(x)) \\
&= w^t_i \text{exp}(y_i( \frac{1}{2} \log \frac{1-\epsilon_t}{\epsilon_t} ) \\
&= w^t_i \text{exp}(y_i( \frac{1}{2} \log \frac{\widetilde{\epsilon}_t}{1- \widetilde{\epsilon}_t} ) \widetilde{h}_t(x)) \\
&= w^t_i \text{exp}(y_i( \frac{1}{2} \log \frac{1- \widetilde{\epsilon}_t}{\widetilde{\epsilon}_t}^{-1} ) \widetilde{h}_t(x))  \\
&= w^t_i \text{exp}(y_i( - \frac{1}{2} \log \frac{1- \widetilde{\epsilon}_t}{\widetilde{\epsilon}_t} ) \widetilde{h}_t(x)) \\
&= w^t_i \text{exp}(- y_i\widetilde{\beta}_t \widetilde{h}_t(x)) 
\end{align*}
Thus showing how the updates are equivalent to each other.\\\\\\
\textbf{Q1d)} To begin we are given the following minimizer:
\[ \text{min}_{\beta} E[e^{-y\beta h_t(x)} |X=x] \]
From this we can then derive that:
\[ = \text{min}_{\beta} \sum_i^n p_i \times e^{-y\beta h_t(x)} \]
Note that we have two cases, either the data is correctly classified in which case $-yh_t(x)$ is negative or the case where the data is classified incorrectly such that $-yh_t(x)$ is positive. Thus we can split this into:
\[  = \text{min}_{\beta} \sum_{i, y=h_t(x)}^n p_i \times e^{-y\beta h_t(x)} +  \sum_{i, y\neq h_t(x)}^n p_i \times e^{y\beta h_t(x)} \]
Another way we can write this is to use the [[$h_t \neq y_i$]] notation as its equivalent: 
\[  = \text{min}_{\beta} \sum_{i}^n p_i \times e^{-\beta} \times [[h_t = y_i]] +  \sum_{i}^n p_i \times e^{\beta} \times [[h_t \neq y_i] \]
\[  = \text{min}_{\beta}  e^{\beta} \sum_{i}^n p_i \times [[h_t = y_i]] +  e^{\beta} \sum_{i}^n p_i \times [[h_t \neq y_i] \]
From the definition given we $\sum_{i}^n p_i \times [[h_t = y_i]] = 1 - \epsilon_t$ and $\sum_{i}^n p_i \times [[h_t \neq y_i]] = \epsilon_t$ and so we get:
\begin{align*}
&= \text{min}_{\beta}  e^{-\beta} \sum_{i}^n p_i \times [[h_t = y_i]] +  e^{\beta} \sum_{i}^n p_i \times [[h_t \neq y_i] \\
&= \text{min}_{\beta}  e^{-\beta} (1-\epsilon_t) +  e^{\beta} \epsilon_t
\end{align*}
Taking the derivative w.r.t $\beta$ gives us:
\begin{align*}
0 &=  e^{-\beta} (1-\epsilon_t) +  e^{\beta} \epsilon_t \\
0 &=  -e^{-\beta} + e^{-\beta}\epsilon_t +  e^{\beta} \epsilon_t \\
0 &= \epsilon_t - 1 + e^{2\beta} \epsilon_t \\
1 - \epsilon_t &= e^{2\beta} \epsilon_t \\
\frac{1 - \epsilon_t}{\epsilon_t} &= e^{2\beta} \\
\beta &= \frac{1}{2} \log \frac{1-\epsilon_t}{\epsilon_t}
\end{align*}
Showing that our $\beta$ is optimal as required. \\\\
\textbf{Q1e)} We are given by equation 10), that:
\end{titlepage}
\end{document}