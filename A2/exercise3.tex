\documentclass{article}
\usepackage[margin=1in]{geometry}
\usepackage{amsmath,amsfonts,amssymb}
\usepackage{listings}
\usepackage{color}
\usepackage{graphicx}
\usepackage{subfig}
\usepackage{blkarray}
\usepackage{multirow}
\usepackage{float}
\usepackage{caption}
\usepackage{subcaption}
\begin{document}
\begin{titlepage}
	\setlength{\parindent}{0pt}
	\large

\vspace*{-2cm}

\definecolor{dkgreen}{rgb}{0,0.6,0}
\definecolor{gray}{rgb}{0.5,0.5,0.5}
\definecolor{mauve}{rgb}{0.58,0,0.82}

\lstset{frame=tb,
  language=Python,
  aboveskip=3mm,
  belowskip=3mm,
  showstringspaces=false,
  columns=flexible,
  basicstyle={\small\ttfamily},
  numbers=none,
  numberstyle=\tiny\color{gray},
  keywordstyle=\color{blue},
  commentstyle=\color{dkgreen},
  stringstyle=\color{mauve},
  breaklines=true,
  breakatwhitespace=true,
  tabsize=3
}

University of Waterloo \par
CS 480 \par
\vspace{0.05cm}
r2knowle: 2023-10-22
\vspace{0.2cm}

{\huge Exercise \# 3 \par}
\hrule

\vspace{0.5cm}
\textbf{Q1)} To begin we are given the kernel function:
\[ k(x,y) = e^{-\alpha(x-y)^2} \]
To find the feature map we will start by converting it into its Taylor series approximation: 
\begin{align*}
e^{-\alpha(x-y)^2} &= \sum^\infty_{k=0}\frac{-\alpha^k(x-y)^2k}{k!} \\
&= \sum^\infty_{k=0}\frac{-\alpha^k}{k!}(x-y)^2k \\
&= \sum^\infty_{k=0}\frac{-\alpha^k}{k!}(x^2-2xy+y^2)^k \\
&= \sum^\infty_{k=0}(\frac{(-\alpha x^2)^k}{k!} + \frac{(-2\alpha xy)^k}{k!} + \frac{(-\alpha y^2)^k}{k!} ) \\
&= \sum^\infty_{k=0}(\frac{(-\alpha x^2)^k}{k!} + \frac{(\sqrt{-2\alpha})^kx^k}{\sqrt{k!}}\frac{(\sqrt{-2\alpha})^ky^k}{\sqrt{k!}} + \frac{(-\alpha y^2)^k}{k!} ) \\ 
<\phi(x), \phi(y)>&= e^{(-\alpha x^2)} \dot e^{(-\alpha x^2)} \sum^\infty_{k=0}\frac{(\sqrt{-2\alpha})^kx^k}{\sqrt{k!}}\frac{(\sqrt{-2\alpha})^ky^k}{\sqrt{k!}}
\end{align*}
Thus we get the feature mappings of:
\[ \phi(x)  = e^{(-\alpha x^2)} \left[1, \frac{\sqrt{-2\alpha}x}{\sqrt{1}}, \frac{\sqrt{-2\alpha}x^2}{\sqrt{2}}, ...  \right]^t \] 
\[ \phi(y)  = e^{(-\alpha y^2)} \left[1, \frac{\sqrt{-2\alpha}y}{\sqrt{1}}, \frac{\sqrt{-2\alpha}y^2}{\sqrt{2}}, ...  \right] \] 
We would want to use the primal in SVM, as the feature space for the kernel is infinite whereas the primal it is not. \\\\
\textbf{Q2)} Since $x,y \in (-1, 1)$, it follows that $|xy| < 1$ Because of this we can use the taylor series expansions to get that:
\begin{align*}
\frac{1}{1-xy} = \sum^\infty_{k=0}(xy)^k
\end{align*}
This will give us the feature maps of:
\[ \phi(x) = \left[1, x, x^2, x^3, ... \right]^T \]
\[ \phi(y) = \left[1, y, y^2, y^3, ... \right] \]
\newpage
\textbf{Q3)} This is not a valid kernel, consider x = 12 and y = 1. Thus we get that:
\[ M = \begin{bmatrix}
\log(145) & \log(13) \\
\log(13) & \log(2)
\end{bmatrix} \]
If we then consider the vector $v = \begin{bmatrix} 1 \\
-6
\end{bmatrix}$ when we multiply we will get:
\begin{align*}
\begin{bmatrix} 1 &
-6
\end{bmatrix} \begin{bmatrix}
\log(145) & \log(13) \\
\log(13) & \log(2)
\end{bmatrix} \begin{bmatrix} 1 \\
-6
\end{bmatrix} &= \begin{bmatrix} \log(145) - 6\log(13) &
\log(13) - 6\log(2)
\end{bmatrix} \begin{bmatrix} 1 \\
-6
\end{bmatrix} \\
&= \log(145)+36\log(2)-12\log(13) \\
&\approx -0.37
< 0
\end{align*}
Thus since its not positive semi-definite it cant be a valid kernel.\\\\
\textbf{Q4)} This is not a valid kernel, consider x = 1 and y = 2. Thus we get that:
\[ M = \begin{bmatrix}
\cos(2) & \cos(3) \\
\cos(3) & \cos(4)
\end{bmatrix} \]
If we then consider the vector $v = \begin{bmatrix} 1 \\
1
\end{bmatrix}$ when we multiply we will get:
\begin{align*}
\begin{bmatrix} 1 &
1
\end{bmatrix} \begin{bmatrix}
\cos(2) & \cos(3) \\
\cos(3) & \cos(4)
\end{bmatrix} \begin{bmatrix} 1 \\
1
\end{bmatrix} &= \begin{bmatrix} \cos(2) + \cos(3) &
\cos(3) + \cos(4)
\end{bmatrix} \begin{bmatrix} 1 \\
1
\end{bmatrix} \\
&= \cos(2) + 2\cos(3) + \cos(4) \\
&\approx -3.05
< 0
\end{align*}
Thus since its not positive semi definite it cant be a valid kernel.
\end{titlepage}
\end{document}